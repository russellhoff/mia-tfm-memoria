\documentclass[12pt]{article}

%Paquetes
\usepackage{titlesec}
\usepackage{xcolor}
\usepackage{lipsum}
\usepackage{fontspec}
\usepackage{graphicx} %Imágenes
\usepackage{colortbl}
\usepackage{setspace}
\usepackage[a4paper]{geometry}
\usepackage{fancyhdr}
\usepackage[babel]{csquotes}
\usepackage[spanish, english]{babel}
\usepackage{apacite} % Norma APA bibliografía
\usepackage{natbib} %Bibliografía
\usepackage[nottoc]{tocbibind}
\usepackage[acronym,nonumberlist,toc]{glossaries} %Configuraciones glosario
\usepackage{glossary-superragged} %Configuraciones glosario
\usepackage[hang,flushmargin]{footmisc}
\usepackage{etoolbox}
\usepackage[hidelinks, breaklinks=true]{hyperref}
\usepackage{booktabs} % Tablas
\usepackage{tabularx}
\usepackage{float}
\usepackage{hyperref}

\usepackage{listings}
\usepackage{color}

\definecolor{dkgreen}{rgb}{0,0.6,0}
\definecolor{gray}{rgb}{0.5,0.5,0.5}
\definecolor{mauve}{rgb}{0.58,0,0.82}

\lstset{frame=tb,
	language=Python,
	aboveskip=3mm,
	belowskip=3mm,
	showstringspaces=false,
	columns=flexible,
	basicstyle={\small\ttfamily},
	numbers=none,
	numberstyle=\tiny\color{gray},
	keywordstyle=\color{blue},
	commentstyle=\color{dkgreen},
	stringstyle=\color{mauve},
	breaklines=true,
	breakatwhitespace=true,
	tabsize=3
}



%Variables
\definecolor{gray80}{gray}{.80}
\definecolor{blueUnir}{HTML}{0098CD}

\geometry{top=2.5cm, bottom=2.5cm, left=3.0cm, right=2.0cm}
\setmainfont{Calibri}
\spacing{1.5} %Interlineado fijo
\setlength{\parskip}{6pt} %6 puntos de espaciado entre párrafos
\setlength{\parindent}{0cm} %Eliminar sangría
\setlength{\footnotesep}{0pt} %Espaciado entre notas
\setlength{\skip\footins}{1.5cm} %Espaciado entre raya y texto
\renewcommand{\footnotelayout}{\small\baselineskip=10pt} % Interlineado sencillo

\fancyhf{}
\pagestyle{fancy}
\rhead[\fontsize{10pt}{12pt}\setmainfont{Calibri Light}\selectfont Jon Inazio Sánchez Martínez\\Predicción de tráfico mediante aprendizaje profundo y Transformers]{\fontsize{10pt}{12pt}\setmainfont{Calibri Light}\selectfont Jon Inazio Sánchez Martínez\\Predicción de tráfico mediante aprendizaje profundo y Transformers} 
\renewcommand{\headrulewidth}{0pt}
%\renewcommand{\footrulewidth}{1pt}
\rfoot[]{\thepage}
\setcounter{tocdepth}{3} 
\setcounter{secnumdepth}{5}
\newcommand\fh{\babelhyphen{hard}}



\titleformat*{\section}{\fontsize{18pt}{18}\selectfont\color{blueUnir}\setmainfont{Calibri Light}} 
\titleformat*{\subsection}{\fontsize{14pt}{14}\selectfont\color{blueUnir}\setmainfont{Calibri Light}} 
\titleformat*{\subsubsection}{\fontsize{12pt}{12}\selectfont\setmainfont{Calibri}\bfseries}

%
% Acrónimos
%
% La forma de definir un acrónimo es la siguiente:
% \newacronym{id}{siglas}{descripción}
% Donde:
% 	'id' es como vas a llamarlo desde el documento.
%	'siglas' son las siglas del acrónimo.
%	'descripción' es el texto que representan las siglas.
%
% Para usarlo en el documento tienes 4 formas:
% \gls{id} - Añade el acrónimo en su forma larga y con las siglas si es la primera vez que se utiliza, el resto de veces solo añade las siglas. (No utilices este en títulos de capítulos o secciones).
% \glsentryshort{id} - Añade solo las siglas de la id
% \glsentrylong{id} - Añade solo la descripción de la id
% \glsentryfull{id} - Añade tanto  la descripción como las siglas

\newacronym{dl}{DL}{Deep Learning o Aprendizaje Profundo}
\newacronym{its}{ITS}{Sistemas Inteligentes de Transporte}
\newacronym{capv}{CAPV}{Comunidad Autónoma del País Vasco}
\newacronym{rnn}{RNN}{Recurrent Neural Networks o Redes Neuronales Recurrentes}
\newacronym{cnn}{CNN}{Convolutional Neural Networks o Redes Neuronales Convolucionales}
\newacronym{lstm}{LSTM}{Long Short-Term Memory}
\newacronym{gnn}{GNN}{Graph Neural Networks}
\newacronym{gru}{GRU}{Gated Recurrent Units}
\newacronym{gcn}{GCN}{Graph Convolutional Networks}
\newacronym{ggnn}{GGNN}{Gated Graph Neural Networks}
\newacronym{gat}{GAT}{Graph Attention Networks}
\newacronym{mlp}{MLP}{Multi Layer Perceptron o Perceptrones Multi Capa}
\newacronym{rmse}{RMSE}{Root Mean Square Error o Error Cuadrático Medio}
\newacronym{mape}{MAPE}{Mean Absolute Percentage Error o Error Porcentual Absoluto Medio}
\newacronym{mae}{MAE}{Mean Absolute Error o Error Medio Absoluto}
\newacronym{mre}{MRE}{Mean Relative Error o Error Medio Relativo}
\newacronym{svr}{SVR}{Support Vector Regression}
\newacronym{svm}{SVM}{Support Vector Machines}
\newacronym{rf}{RF}{Random Forests}
\newacronym{knn}{KNN}{K-Nearest Neighbors}
\newacronym{arima}{ARIMA}{AutoRegressive Integrated Moving Average}
\newacronym{sarima}{SARIMA}{Seasonal AutoRegressive Integrated Moving Average}
\newacronym{gpu}{GPU}{Graphics Processing Unit}
\newacronym{ram}{RAM}{Random Access Memory}
\newacronym{ssd}{SSD}{Solid State Drive}

%
% Glosario
%
%\newglossaryentry{latex}
%{
	%	name=latex,
	%	description={Is a mark up language specially suited for scientific documents}
	%}
\newglossaryentry{api}{
	name=API,
	description={Interfaz de programación de aplicaciones. Conjunto de funciones y definiciones que permiten la comunicación entre sistemas de software}
}

\newglossaryentry{json}{
	name=JSON,
	description={JavaScript Object Notation. Formato ligero y estructurado de intercambio de datos, ampliamente utilizado en APIs y configuraciones}
}

\newglossaryentry{geojson}{
	name=GeoJSON,
	description={Extensión del formato JSON para representar objetos geoespaciales como puntos, líneas, polígonos o colecciones de geometrías}
}

\newglossaryentry{uml}{
	name=UML,
	description={Unified Modeling Language. Lenguaje de modelado visual estandarizado para representar sistemas software desde distintas perspectivas (estructural, de comportamiento, etc.)}
}

\newglossaryentry{nosql}{
	name=NoSQL,
	description={Modelo de bases de datos no relacional orientado a documentos, grafos, columnas o pares clave-valor, ideal para sistemas distribuidos, escalables y flexibles}
}

\newglossaryentry{arquitectura-hexagonal}{
	name=arquitectura hexagonal,
	description={Estilo de diseño de software que promueve la separación de la lógica del dominio respecto a la infraestructura, facilitando el mantenimiento, las pruebas y la escalabilidad}
}

\newglossaryentry{aforo}{
	name=aforo,
	description={Medida del flujo de vehículos que pasan por un punto determinado de una carretera o vía en un periodo de tiempo concreto. Utilizado para analizar la intensidad y distribución del tráfico.}
}

\newglossaryentry{redviaria}{
	name=red viaria,
	description={Conjunto de infraestructuras y vías (carreteras, autopistas, calles) que conforman la red de transporte terrestre de una región o país.}
}

\newglossaryentry{aforador}{
	name=aforador,
	description={Dispositivo utilizado para medir el número de vehículos que pasan por un punto de la red viaria, registrando el flujo de tráfico}
}

\newglossaryentry{resampling}{
	name=resampling,
	description={Proceso de agregación o interpolación de datos temporales para ajustarse a una nueva frecuencia}
}

\newglossaryentry{ventana}{
	name=ventana deslizante,
	description={Técnica que permite recorrer una serie temporal generando subconjuntos de datos consecutivos}
}

\newglossaryentry{normalizacion}{
	name=normalización,
	description={Transformación de los datos para que tengan media cero y desviación estándar uno u otro rango definido}
}

\newglossaryentry{onnx}{
	name=ONNX,
	description={Open Neural Network Exchange. Formato abierto y estandarizado para representar modelos de aprendizaje automático, diseñado para facilitar la interoperabilidad entre diferentes frameworks de deep learning como PyTorch, TensorFlow, Keras o Scikit-learn}
}


\newglossaryentry{design_pattern:builder}{
	name=Builder,
	description={Separa la construcción de un objeto complejo de su representación, permitiendo que el mismo proceso de construcción pueda crear diferentes representaciones. Utilizado en la generación del dataset a través de la clase \texttt{MobilitySnapshotBuilder}}
}
\newglossaryentry{design_pattern:repository}{
	name=Repository,
	description={Media entre el dominio y las capas de mapeo de datos, proporcionando una interfaz similar a una colección para acceder a los objetos del dominio. Todas las operaciones de acceso a datos (\texttt{FlowRepository}, \texttt{MeterRepository}, etc.) se abstraen en repositorios desacoplados}
}
\newglossaryentry{design_pattern:servicelayer}{
	name=Service Layer,
	description={Define los límites de una aplicación y las operaciones disponibles, coordinando la respuesta de la aplicación en cada operación. La lógica de negocio reside en servicios como \texttt{FlowService}, \texttt{MeterService}, \texttt{IncidenceService}}
}
\newglossaryentry{design_pattern:dto}{
	name=Data Transfer Object,
	description={Es un objeto que transporta datos entre procesos, generalmente para reducir el número de llamadas a métodos remotos. Se emplean DTOs para el intercambio de datos entre capas, evitando el acoplamiento con los modelos de persistencia.}
}
\newglossaryentry{design_pattern:helper_utility}{
	name=Helper/Utility,
	description={Helper: Proporciona funcionalidad reutilizable y de bajo nivel a otras clases, a menudo a través de métodos estáticos, para evitar la duplicación de código. Utility: Es una clase que agrupa métodos estáticos con funcionalidades comunes y reutilizables que no dependen del estado de ningún objeto. Utilidades para parseo, validación y transformación de datos (por ejemplo, para el tratamiento de ficheros XML meteorológicos)}
}
\newglossaryentry{design_pattern:facade}{
	name=Facade,
	description={Ofrece una interfaz unificada y simplificada a un conjunto de interfaces en un subsistema más complejo, facilitando su uso. Fachadas que agrupan operaciones complejas en interfaces sencillas, facilitando la integración con servicios externos.}
}
\newglossaryentry{design_pattern:factory_singleton}{
	name=Factory/Singleton,
	description={Factory: Se encarga de crear objetos sin especificar la clase exacta del objeto que se creará, delegando esta lógica a una subclase. Garantiza que una clase tenga una única instancia y proporciona un punto de acceso global a ella. Empleados para instanciar objetos según el origen de datos y para servicios centrales, respectivamente.}
}
\makeglossaries

\makeatletter
\patchcmd{\@footnotetext}{\footnotesize}{\fontsize{10pt}{12pt}\setmainfont{Calibri}}{}{}
\makeatother

\addto\captionsspanish{%
	\renewcommand*\contentsname{Índice de contenidos}
	\renewcommand{\listtablename}{Índice de tablas} 
	\renewcommand{\tablename}{Tabla} 
	%\renewcommand{\bibname}{Referencia bibliográfica}
}

\titlespacing*{\paragraph}{0pt}{9pt}{0.5ex}
\titleformat{\paragraph}[block]{\normalsize}{\theparagraph}{.5em}{\mdseries}
\titlespacing*{\subparagraph}{0pt}{9pt}{0.5ex}
\titleformat{\subparagraph}[block]{\normalsize}{\thesubparagraph}{.5em}{\mdseries}

\renewenvironment{description}
{\list{}{\labelwidth0pt\itemindent-\leftmargin\parsep0pt\itemsep0pt\let\makelabel\descriptionlabel}}{\endlist}