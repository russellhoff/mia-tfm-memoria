%
% Acrónimos
%
% La forma de definir un acrónimo es la siguiente:
% \newacronym{id}{siglas}{descripción}
% Donde:
% 	'id' es como vas a llamarlo desde el documento.
%	'siglas' son las siglas del acrónimo.
%	'descripción' es el texto que representan las siglas.
%
% Para usarlo en el documento tienes 4 formas:
% \gls{id} - Añade el acrónimo en su forma larga y con las siglas si es la primera vez que se utiliza, el resto de veces solo añade las siglas. (No utilices este en títulos de capítulos o secciones).
% \glsentryshort{id} - Añade solo las siglas de la id
% \glsentrylong{id} - Añade solo la descripción de la id
% \glsentryfull{id} - Añade tanto  la descripción como las siglas

\newacronym{dl}{DL}{Deep Learning o Aprendizaje Profundo}
\newacronym{its}{ITS}{Sistemas Inteligentes de Transporte}
\newacronym{capv}{CAPV}{Comunidad Autónoma del País Vasco}
\newacronym{rnn}{RNN}{Recurrent Neural Networks o Redes Neuronales Recurrentes}
\newacronym{cnn}{CNN}{Convolutional Neural Networks o Redes Neuronales Convolucionales}
\newacronym{lstm}{LSTM}{Long Short-Term Memory}
\newacronym{gnn}{GNN}{Graph Neural Networks}
\newacronym{gru}{GRU}{Gated Recurrent Units}
\newacronym{gcn}{GCN}{Graph Convolutional Networks}
\newacronym{ggnn}{GGNN}{Gated Graph Neural Networks}
\newacronym{gat}{GAT}{Graph Attention Networks}
\newacronym{mlp}{MLP}{Multi Layer Perceptron o Perceptrones Multi Capa}
\newacronym{rmse}{RMSE}{Root Mean Square Error o Error Cuadrático Medio}
\newacronym{mape}{MAPE}{Mean Absolute Percentage Error o Error Porcentual Absoluto Medio}
\newacronym{mae}{MAE}{Mean Absolute Error o Error Medio Absoluto}
\newacronym{mre}{MRE}{Mean Relative Error o Error Medio Relativo}
\newacronym{svr}{SVR}{Support Vector Regression}
\newacronym{svm}{SVM}{Support Vector Machines}
\newacronym{rf}{RF}{Random Forests}
\newacronym{knn}{KNN}{K-Nearest Neighbors}
\newacronym{arima}{ARIMA}{AutoRegressive Integrated Moving Average}
\newacronym{sarima}{SARIMA}{Seasonal AutoRegressive Integrated Moving Average}
\newacronym{gpu}{GPU}{Graphics Processing Unit}
\newacronym{ram}{RAM}{Random Access Memory}
\newacronym{ssd}{SSD}{Solid State Drive}

%
% Glosario
%
%\newglossaryentry{latex}
%{
	%	name=latex,
	%	description={Is a mark up language specially suited for scientific documents}
	%}
\newglossaryentry{api}{
	name=API,
	description={Interfaz de programación de aplicaciones. Conjunto de funciones y definiciones que permiten la comunicación entre sistemas de software}
}

\newglossaryentry{json}{
	name=JSON,
	description={JavaScript Object Notation. Formato ligero y estructurado de intercambio de datos, ampliamente utilizado en APIs y configuraciones}
}

\newglossaryentry{geojson}{
	name=GeoJSON,
	description={Extensión del formato JSON para representar objetos geoespaciales como puntos, líneas, polígonos o colecciones de geometrías}
}

\newglossaryentry{uml}{
	name=UML,
	description={Unified Modeling Language. Lenguaje de modelado visual estandarizado para representar sistemas software desde distintas perspectivas (estructural, de comportamiento, etc.)}
}

\newglossaryentry{nosql}{
	name=NoSQL,
	description={Modelo de bases de datos no relacional orientado a documentos, grafos, columnas o pares clave-valor, ideal para sistemas distribuidos, escalables y flexibles}
}

\newglossaryentry{arquitectura-hexagonal}{
	name=arquitectura hexagonal,
	description={Estilo de diseño de software que promueve la separación de la lógica del dominio respecto a la infraestructura, facilitando el mantenimiento, las pruebas y la escalabilidad}
}

\newglossaryentry{aforador}{
	name=aforador,
	description={Dispositivo utilizado para medir el número de vehículos que pasan por un punto de la red viaria, registrando el flujo de tráfico}
}

\newglossaryentry{resampling}{
	name=resampling,
	description={Proceso de agregación o interpolación de datos temporales para ajustarse a una nueva frecuencia}
}

\newglossaryentry{ventana}{
	name=ventana deslizante,
	description={Técnica que permite recorrer una serie temporal generando subconjuntos de datos consecutivos}
}

\newglossaryentry{normalizacion}{
	name=normalización,
	description={Transformación de los datos para que tengan media cero y desviación estándar uno u otro rango definido}
}
% API
% Aforador
% JSON
% GeoJSON
% Arquitectura hexagonal
% UML
% NoSQL
