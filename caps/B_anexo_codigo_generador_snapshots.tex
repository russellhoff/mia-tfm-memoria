\section*{Anexo B – Código fuente del generador de snapshots}
\label{anexo:snapshot_generator}
\addcontentsline{toc}{section}{Anexo B – Código fuente del generador de snapshots}

A continuación se presenta el código fuente completo de la clase \texttt{MobilitySnapshotGeneratorService}, responsable de generar las instancias de \texttt{MobilitySnapshot} mediante la integración de datos de tráfico, meteorología e incidencias.

\begin{lstlisting}[language=Kotlin, caption={Clase MobilitySnapshotGeneratorService}]
	package es.joninx.tfm.dc.service
	
	import es.joninx.tfm.dc.builder.dataset.MobilitySnapshotBuilder
	import es.joninx.tfm.dc.config.Cfg
	import es.joninx.tfm.dc.repository.meteo.ReadingXmlRepository
	import es.joninx.tfm.dc.repository.meteo.StationRepository
	import es.joninx.tfm.dc.repository.traffic.FlowRepository
	import es.joninx.tfm.dc.repository.traffic.IncidenceRepository
	import es.joninx.tfm.dc.repository.traffic.MeterRepository
	import es.joninx.tfm.dc.util.DateTimeUtils
	import org.apache.logging.log4j.LogManager
	import org.apache.logging.log4j.Logger
	import org.springframework.stereotype.Service
	import java.time.Duration
	import java.time.LocalDateTime
	
	@Service
	class MobilitySnapshotGeneratorService(
		private val cfg: Cfg,
	
		private val snapshotBuilder: MobilitySnapshotBuilder,
		private val snapshotPersistenceService: MobilitySnapshotPersistenceService,
		private val flowRepository: FlowRepository,
		private val meterRepository: MeterRepository,
		private val incidenceRepository: IncidenceRepository,
		private val stationRepository: StationRepository,
		private val meteoReadingsRepository: ReadingXmlRepository,
	) {
		
		/**
		* Mapa cache: meterId → stationId
		*/
		private val meterToStationCache = mutableMapOf<String, String>()
		
		fun getNearestStationId(meterId: String, latitude: Double, longitude: Double): String? {
			// ¿Ya lo tenemos cacheado?
			meterToStationCache[meterId]?.let {
				log.debug("Cache HIT: meterId=$meterId → stationId=$it")
				return it
			}
			
			// Si no, buscamos la estación más cercana
			val nearestStation = stationRepository.findNearestStation(
			longitude = longitude,
			latitude = latitude,
			maxDistanceMeters = cfg.algorithm.maxDistanceToStation
			).blockFirst() ?: return null
			
			meterToStationCache[meterId] = nearestStation.stationId
			log.debug("Cache MISS: meterId=$meterId → stationId=${nearestStation.stationId}")
			return nearestStation.stationId
		}
		
		fun generateSnapshots(
			sourceIds: List<String>,
			startDate: LocalDateTime,
			endDate: LocalDateTime,
			batchSize: Int = 500
		) {
			log.debug("Comenzando generación de MobilitySnapshots para sourceIds=${sourceIds.joinToString(",")}, fechas entre '${DateTimeUtils.format(startDate)}' y '${DateTimeUtils.format(endDate)}'")
			
			val meters = meterRepository.findAllBySourceIdIn(sourceIds)
			.collectMap { it.meterId }
			.block() ?: emptyMap()
			
			val intervals = generateIntervals(startDate, endDate, cfg.algorithm.timeWindowDuration)
			var totalSnapshots = 0
			
			intervals.forEachIndexed { idx, (intervalStart, intervalEnd) ->
				val flows = flowRepository.findAllBySourceIdInAndDateTimeBetweenQuery(
				sourceIds, intervalStart, intervalEnd
				).collectList().block() ?: emptyList()
				
				val groupedByMeter = flows.groupBy { it.meterId }
				
				val snapshots = groupedByMeter.mapNotNull { (meterId, flowList) ->
					val meter = meters[meterId]
					if (meter != null && flowList.isNotEmpty()) {
						val totalVehiclesSum = flowList.sumOf { it.totalVehicles.toIntOrNull() ?: 0 }
						
						// Obtener incidencias para el intervalo
						val latitude = meter.latitude
						val longitude = meter.longitude
						
						// Busca incidencias cercanas y activas
						val incidences = incidenceRepository
						.findIncidencesNearAndActive(
						longitude, latitude, cfg.algorithm.maxDistanceToIncidences,
						intervalStart, intervalEnd
						)
						.collectList()
						.block() ?: emptyList()
						
						// Meteo
						val nearestStationId = getNearestStationId(
						meterId = meterId,
						latitude = latitude,
						longitude = longitude
						)
						val meteoReadings = if (nearestStationId != null) {
							meteoReadingsRepository.findReadingsByStationIdAndDateTimeBetween(
							stationId = nearestStationId,
							windowStart = intervalStart,
							windowEnd = intervalEnd
							).collectList().block() ?: emptyList()
						} else emptyList()
						
						log.debug("Intervalo '${DateTimeUtils.format(intervalStart)}' → '${DateTimeUtils.format(intervalEnd)}' | meterId=$meterId | numFlows=${flowList.size} | totalVehiclesSum=$totalVehiclesSum | totalIncidences=${incidences.size} |meteoReadings=${meteoReadings.size}")
						snapshotBuilder.fromGroupedFlows(
						flows = flowList,
						meter = meter,
						windowStartDateTime = intervalStart,
						windowEndDateTime = intervalEnd,
						totalVehicles = totalVehiclesSum,
						incidences = incidences,
						meteoReadings = meteoReadings,
						)
					} else {
						if (meter == null) {
							log.warn("MeterId=$meterId no encontrado en meterMap para ventana '${DateTimeUtils.format(intervalStart)}' → '${DateTimeUtils.format(intervalEnd)}'")
						} else {
							log.warn("No hay flows para meterId=$meterId en ventana '${DateTimeUtils.format(intervalStart)}' → '${DateTimeUtils.format(intervalEnd)}'")
						}
						null
					}
				}
				
				// Guardar por lotes
				snapshots.chunked(batchSize).forEach { batch ->
					snapshotPersistenceService.saveBatch(batch).block()
					log.debug("Batch guardado para intervalo $intervalStart → $intervalEnd (${batch.size} snapshots)")
				}
				totalSnapshots += snapshots.size
				
				if ((idx + 1) % 24 == 0) { // Cada 12 horas
					log.debug("Progreso: ${idx + 1} de ${intervals.size} intervalos procesados, $totalSnapshots snapshots generados.")
				}
			}
			
			log.debug("Generación de MobilitySnapshots finalizada. Total snapshots: $totalSnapshots")
		}
		
		// Utilidad para generar ventanas de 30 minutos (la misma que antes)
		fun generateIntervals(start: LocalDateTime, end: LocalDateTime, step: Duration): List<Pair<LocalDateTime, LocalDateTime>> {
			val intervals = mutableListOf<Pair<LocalDateTime, LocalDateTime>>()
			var current = start
			while (current.isBefore(end)) {
				val next = current.plus(step)
				intervals.add(Pair(current, if (next.isBefore(end)) next else end))
				current = next
			}
			return intervals
		}
		
		companion object {
			val log: Logger = LogManager.getLogger(this::class.java)
		}
		
	}
\end{lstlisting}