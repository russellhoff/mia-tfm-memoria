\section*{Anexo A – Direcciones a los repositorios del TFM}
\label{anexo:sensores}
\addcontentsline{toc}{section}{Anexo A – Direcciones a los repositorios del TFM}

En este anexo se incluyen las direcciones a los principales repositorios públicos generados y empleados a lo largo del Trabajo Fin de Máster. El objetivo es aportar transparencia y facilitar la reproducibilidad de los resultados y desarrollos presentados.

El repositorio principal de la memoria del TFM está disponible en \url{https://github.com/russellhoff/mia-tfm-memoria}. En él se encuentra el código fuente en \LaTeX{}, las imágenes, anexos, así como todos los recursos necesarios para compilar y reproducir íntegramente este documento.

El proyecto \texttt{Data Collector} se aloja en \url{https://github.com/russellhoff/mia-tfm-data-collector}. Este repositorio contiene el código fuente (Kotlin + Spring Boot) y la documentación del artefacto encargado de la recopilación y consolidación de los datos meteorológicos y de tráfico, necesarios para la generación del dataset base del TFM.

El desarrollo principal de modelos predictivos, junto con el código, notebooks y scripts utilizados para el entrenamiento y evaluación de las arquitecturas (MLP y Trafficformer), se encuentra en el repositorio de \texttt{Trafficformer Bizkaia}: \url{https://github.com/russellhoff/mia-tfm-trafficformer-bizkaia}. En él se documenta todo el proceso de preprocesamiento, entrenamiento y validación de los modelos desarrollados en Python.

Finalmente, se ha empleado la plataforma Weights \& Biases para la gestión y trazabilidad de los experimentos realizados. El proyecto, accesible en \url{https://wandb.ai/jonin/trafficformer-tfm}, recoge los registros de entrenamientos, métricas, visualizaciones, configuración de hiperparámetros y permite comparar de forma interactiva los diferentes resultados obtenidos.

\vspace{0.5cm}

Estos recursos permiten al lector consultar el código, reproducir los experimentos y acceder tanto a los datos como a los modelos y documentación generados a lo largo del trabajo.