\vspace*{\fill}
\selectlanguage{spanish}
\begin{abstract}
	La predicción precisa del tráfico es clave para optimizar la movilidad y el uso de infraestructuras, especialmente en la \acrlong{capv} y más en concreto en la provincia de Bizkaia, donde la orografía, densidad urbana y condiciones meteorológicas suponen retos añadidos. Sin embargo, la complejidad inherente a los datos de tráfico, caracterizados por fuertes correlaciones espacio-temporales y patrones no lineales, dificulta su modelado y predicción exacta. Este trabajo propone un modelo de predicción basado en redes neuronales con arquitectura Transformer, capaz de capturar dependencias espacio-temporales complejas mediante mecanismos de atención que asignan pesos dinámicos a segmentos relevantes. Se emplearán datos abiertos de tráfico y meteorología procedentes de la provincia de Bizkaia. La validación se realizará con datos reales y se comparará frente a modelos tradicionales de redes neuronales, demostrando mejoras significativas en precisión. El sistema resultante ofrece una herramienta robusta y eficiente para la gestión del tráfico en tiempo real, con potencial para anticipar congestiones y optimizar la toma de decisiones operativas.
\end{abstract}
\textbf{Palabras clave}: Predicción del tráfico, Aprendizaje profundo, Transformers, Redes neuronales, Correlación espacio-temporal, Datos abiertos, Comunidad Autónoma del País Vasco, Bizkaia

\vfill
\clearpage
\vspace*{\fill}

\selectlanguage{english}

\begin{abstract}
	Accurate traffic forecasting is essential for optimising mobility and infrastructure usage, especially in the Autonomous Community of the Basque Country (\acrshort{capv}) and particularly in the province of Bizkaia, where the region’s orography, urban density, and weather variability present additional challenges. However, the inherent complexity of traffic data, characterised by strong spatio-temporal correlations and non-linear patterns, makes its modelling and prediction difficult. This work proposes a prediction model based on neural networks with Transformer architecture, capable of capturing complex spatio-temporal dependencies through attention mechanisms that dynamically assign weights to relevant segments. Bizkaia's open traffic and weather datasets will be used. The model will be validated with real-world data and compared against traditional neural network models. The outcome will be a robust and efficient tool for real-time traffic management, with the potential to anticipate congestion and support the optimisation of operational decision-making.
\end{abstract}

\textbf{Keywords}: Traffic forecasting, Deep learning, Transformers, Neural networks, Spatio-temporal correlation, Open data, Autonomous Community of the Basque Country, Bizkaia
\selectlanguage{spanish}
\vspace*{\fill}